\documentclass{article}
\usepackage{geometry}
\usepackage{fancyhdr}
\usepackage{amsmath ,amsthm ,amssymb}
\usepackage{graphicx}
\usepackage{hyperref}
\usepackage{comment}
\usepackage{csvsimple}


\begin{document}

\begin{titlepage}

\newcommand{\HRule}{\rule{\linewidth}{0.5mm}} % Defines a new command for the horizontal lines, change thickness here

\center % Center everything on the page
 
%----------------------------------------------------------------------------------------
%	HEADING SECTIONS
%----------------------------------------------------------------------------------------

%\textsc{\LARGE National University of Singapore}\\[1.5cm] % Name of your university/college
%\textsc{\large Minor Heading}\\[0.5cm] % Minor heading such as course title

%----------------------------------------------------------------------------------------
%	TITLE SECTION
%----------------------------------------------------------------------------------------

\HRule \\[0.4cm]
{ \huge \bfseries CS5340 Project Report}\\[0.4cm] % Title of your document
\HRule \\[1.5cm]


%----------------------------------------------------------------------------------------
%	AUTHOR SECTION
%----------------------------------------------------------------------------------------


% If you don't want a supervisor, uncomment the two lines below and remove the section above
\Large \emph{By:}\\
Neha Priyadarshini Garg (A0117042W)\\ % Your name
Xin Yi (-----)\\
Zhu Lei (A0105790H)\\[3cm]

%----------------------------------------------------------------------------------------
%	DATE SECTION
%----------------------------------------------------------------------------------------

{\large \today}\\[2cm] % Date, change the \today to a set date if you want to be precise

\vfill % Fill the rest of the page with whitespace

\end{titlepage}

\bibliographystyle{apalike}

%\title{Learning and Planning for Autonomous Grasping in Uncontrolled Environments}
%\author{Neha Priyadarshini Garg}
%\maketitle


\section{Problem 1 : Data Denoising}

\section{Problem 2 : Expectation-Maximization Segmentation}


%\bibliography{relatedWork}


\end{document}
